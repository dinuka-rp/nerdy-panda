% lec note: https://docs.google.com/document/d/1mhnFaNrPPQxUFme8OVD2Y1bqOOBA-wp7/edit
% ---------

\section{Chapter Overview}
% 1/4 pg

\section{Problem Domain}
% 3 pgs


\section{Concept Map}
% 1/4 pg + Appendix

\section{Existing work}
% Now it is time to move the Existing work from the proposal to LR (paraphrase it)
% 4/5pgs+-
A hybrid Recommendations System \autocite{cheng_hybrid_2020} which utilizes opinion \& sentiment extraction techniques from user reviews to create preference profiles for movie recommendations, to enhance the quality of recommendations regardless of the rich or sparse nature of the dataset has been identified as one of the recent researches done towards pushing the limits of baseline recommendation models. The framework that has been designed here uses Collaborative Filtering as the base Recommendations model. The contribution of this research is applicable to the feature engineering stage of the system.

Sentiment analysis is applied on user-reviews to detect user-opinions about movies that were watched and reviewed by users. This data is used to create a user's preference profile, similar to what's created in Content-based filtering. The user's sentiment is identified as a step beyond traditional preference ratings.

% advantages
Due to its capability of dealing with insufficient data, the framework is able to produce recommendations that are more accurate and efficient than existing baseline methods. This proves that using public opinion in the feature engineering stage can enhance the quality of recommendations.


% limitations
Due to the fact that the semantic strategy of opinion extraction being generic, it is understood that it may not be ideal to identify different aspects in varied genres. Examples mentioned are, quality of sound may be of greater interest in action movies, while the story-line in dramas.
Slang, irony \& sarcasm haven't been taken into consideration when extracting user opinion.
A major limitation identified in most systems that rely on similar opinion mining systems is that they are very dependant on the text mining technique used. Furthermore, the semantic strategy of extracting user opinion is identified as future work that can be done with regarding to this framework.

% The main drawback that this paper points out is that it

\subsection{Benchmarking}

\section{Review of Different Problem-solving Approaches}
% unused technologies in domain can come here as well.
% 4pgs+-
There are several baseline techniques of Recommendations Systems that have been used by the biggest data-driven companies around the world.
% Explain about Collaborative, Content-based Filtering, Hybrid, Deep Learning Techniques, drawbacks, why baseline methods should be pushed forward using feature engineering, etc.

\section{Review of Evaluation Approaches}
%  (quantitative or qualitative)
% 4pgs+-

\section{Tools}
% 3/4pgs+-

\section{Chapter Summary}
% 1/2 pg