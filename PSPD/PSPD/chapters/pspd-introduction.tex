\section{Chapter Overview}

%  Even though many people have got into purchasing \Gls{nft}s, one of the major problems that owners, as well as potential customers face, is that they’re unable to find NFTs that are worth trading their current NFTs to.
 In this research project, the author tries to identify the required features to be considered for an NFT-trading Recommendations System and introduce a new Ensemble Architecture for Recommendations that can be applied in other related domains as well. The proposed architecture will try to automate several decision-making steps that a user would otherwise need to go through to find the best possible trade.

This chapter defines the problem, the research gap, aims \& objectives of the research and the research challenge that the author wishes to address by completion of the project. The necessary proofs of the problem, as well as previous research interests, are also reviewed.

% Methodologies chapter - Finally, in the Work Plan, the expected schedule of the project's deliverables is presented.

\section{Problem Domain}
\subsection{Non-fungible Tokens (NFTs)}
In recent months, the \Gls{nft} market has been growing exponentially as it appears to be the most widely accepted business application of Blockchain technology \autocite{dowling_is_2021}, since the introduction of crypto.
\Gls{nft}s are provably scarce unique digital assets that can be used to represent ownership \autocite{noauthor_erc-721_nodate}.
They can be one of a kind rare artworks, collectable trading cards, and other assets with the potential to increase in value due to scarcity \autocite{conti_what_2021, fairfield_tokenized_2021}. While being digital assets, they also can be used to represent physical assets. A digital certificate of land/ qualification can be identified as a couple of examples. The biggest winners in the NFT space over the last few months have been digital artists who were able to sell art worth over \$2.5 Billion \parencite{noauthor_off_2021}.


NFTs were introduced by Ethereum \autocite{wood_ethereum_2014} as an improvement proposal \autocite{noauthor_eip-2309_nodate, noauthor_erc_nodate} in the \gls{erc}-721 standard \autocite{noauthor_erc-721_nodate}. This allows anyone to implement a Smart Contract with the ERC-721 standard and let people mint NFTs as well as, keep track of the tokens produced by it. This allows the created tokens to be validated.

% \bigbreak
% Smart Contracts
Each of these created tokens is unique from the other tokens created by the same Smart Contract, unlike fungible tokens which were introduced with cryptocurrencies and are denoted by the ERC-20 standard \autocite{noauthor_erc-20_nodate} on the Ethereum network. One Bitcoin can be swapped to another Bitcoin, but each NFT will be unique.
Then, the deployed Smart Contract will be responsible to keep track of the tokens created by it on the network. A Smart Contract is a program that resides on the Ethereum network with a collection of code \& data \autocite{noauthor_introduction_nodate}.

For each NFT, the contact address \& unit256 tokenId are globally unique on any blockchain. This allows Decentralized Applications (DApps) \autocite{frankenfield_decentralized_nodate, noauthor_decentralized_2021} to take the tokenId and present the image/ asset that is identified by the particular NFT.

% \bigbreak
\emph{"To put it in terms of physical art collecting: anyone can buy a Monet print. But only one person can own the original."} \autocite{clark_people_2021}

While a digital file can be copied regardless of whether it's an NFT or not, what this technology provides is the ownership of the digital asset.
% That is something that cannot be copied or taken from the owner. 
If an NFT that contains your certificate/ domain is held under your wallet on the Blockchain, no one else can get it from you unless they have your digital wallet's private key. Similar to a deed. But, anyone can see, validate and admire what you own.

% ----------------

\subsection{NFT Marketplaces}

% NFT Market places & what they offer. 

% Money in NFT & how markets have expanded (Open sea used by Reddit NFTs) and been funded.
OpenSea, which was the first NFT marketplace is also considered to be the largest. In the attempt to become the "Amazon of NFTs", OepnSea raised \$23 million in a Series A \autocite{hackett_this_2021}, following a \$100 million raise in a Series B round, ended the company in a valuation of \$1.5 billion \autocite{dfinzer_announcing_2021, matney_nft_2021}. Open Sea saw nearly \$150 million in sales in the month of June.
These marketplaces are set to increase access to the digital goods industry \autocite{chevet_blockchain_2018}.


An NFT purchased on an Ethereum marketplace can be traded on any other Ethereum marketplace for a completely different NFT. Creators don't necessarily need to sell their NFT on a market. They can do the transaction peer-to-peer, completely secured by Blockchain. No one is needed to intermediate and an owner isn't locked onto any platform \autocite{noauthor_erc-721_nodate}.

% How they can benefit from Recommendations - add to motivation? - already mentioned in problem

\subsection{Recommendation Systems}
Recommendation Systems have been driving engagement and consumption of content as well as items on almost every corner of the internet over the last decade.
% \bigbreak
These systems help users identify relevant items on an online platform. When users are recommended with relevant items, it enables businesses in growing their revenue. 35\% of Amazon’s revenue \autocite{naumov_deep_2019} \& 60\% of watch time on YouTube \autocite{noauthor_recommendations_nodate} comes from recommendations.


% ***\section{Problem Domain}
% Furthermore, they have to refer to several sources across the internet to find items that are highly trending.



% Although Computer Scientists and Data Scientists have pushed the limits of Recommendation Systems with recent advances in Machine Learning and Deep Learning, it is clear that none of the currently available Recommendation Architectures satisfies the thinking process identified by NFT owners & potential customers when searching for NFTs to trade.


\section{Problem Definition}
Currently, there is no way of identifying possible tradable NFT assets, unless manually browsing through the internet. 
Marketplaces allow searching for NFTs by keywords, categories \& pricing, but don't provide personalized recommendations of trending items.
This applies to someone who wants to purchase an NFT that shows similar characteristics to another NFT that has already been purchased by a previous buyer or oneself. Since there can be only one owner for an NFT at a time, recommendations using standard collaborative filtering is also not entirely ideal. Content-based approaches won't help identify trending items.

\bigbreak
To help with the exploration of these digital assets, it's identified that several steps that the user has to follow to identify trending items that are timely, popular among the community and may have an expected value can be automated. 

% \subsection{Problem Statement}
% It is difficult to find NFTs of comparable value that is trending among the community, timely and relevant to the user’s identified interest or the NFT that the user currently owns.

\section{Research Aims and Objectives}
\subsection{Research Aim}
\textit{The aim of this research is to design, develop \& evaluate a novel Recommendation Architecture that will provide relevant, trending, timely, and worthy NFTs for trading purposes by automating some of the decision making steps that the user would otherwise have to do manually.}

\bigbreak
To elaborate on the aim, this research project will produce a system \& architecture that can be used to recommend trending items with respect to a chosen item in a specific data set. The focus will be laid on the recommendation of NFTs. In order to achieve this several public channels of trends will be required to be streamed into the recommendations architecture together with the automation of several decision-making steps that a user that is interested in purchasing NFTs would have to manually go through, in order to make the best possible trade. The use of Data Mining techniques, \Gls{nlp} techniques, Data Analysis, hybrid, content-based, collaborative filtering \& Deep Learning methods will be researched to make the best possible recommendations.

The required knowledge will be studied and researched, components will be developed and the performance will be evaluated in order to validate or invalidate the chosen hypothesis. The system will be able to run in a local browser for personal use or in a hosted server for public use. The data science models \& their code will be available for further research and use in a public repository that is easy to get up and running with ease. A review paper will be published with knowledge gathered from the survey of Literature. A research paper will be published on the outcome of the findings in the research project.

\subsection{Research Objectives}
The Aims and Research Questions mentioned above are expected to be achieved and answered with the completion of the following Research Objectives. These objectives are milestones that will be expected to be met in order for the research to be completed successfully.

% \newpage

% breakdown all the rows into atomic objectives to be done
% \begin{table}[h!]
% \centering
\begin{longtable}{| p{0.135\linewidth} | p{0.67\linewidth}| p{0.12\linewidth}|}
\caption{Research Objectives}
\label{tab:research-objectives-table}\\
\hline
Objective &   Description & Learning Outcomes  \\ 
\hline
% Problem Identification & Identifying a suitable and valuable problem domain to contribute towards, with identified research gaps suitable for a research project.
% & LO5 \\
% \hline
Literature Survey & Read previous work to collate relevant information on related work and critically evaluate them.
\begin{itemize}
\item \textbf{RO1:} Conduct a preliminary study on existing Recommendations Systems \& Architectures.
\item \textbf{RO2:} Analyze the perception of Recommendation techniques.
\item \textbf{RO3:} Conduct a preliminary study on \Gls{nft}s.
\item \textbf{RO4:} Analyze user desires and factors that affect the likability of owning \Gls{nft}s.
\vspace{-7mm}       % remove line spacing after itemize
\end{itemize}
& LO4, LO2, LO5 \\
\hline
% Project Methodology & Choosing the Research, Development and Project Methodologies that can be followed. Creating a project plan with expected activities and scheduled times for the time frame allocated for the project.
% & LO3, LO7 \\
% \hline
Requirement Analysis &  Specifying the requirements of the project using appropriate techniques and tools in order to meet the expected research gaps \& challenges to be addressed based on previous related research and any domain-specific sources of knowledge.
\begin{itemize}
\item \textbf{RO5:} Gather information about requirements related to desirability of owning \Gls{nft}s \& crypto-related assets.
\item \textbf{RO6:} Gather the requirements of a Recommendations System and understand end-user expectations.
\item \textbf{RO7:} Get insights \&  opinions from technology \& domain experts to build a suitable system.
\vspace{-7mm}       % remove line spacing after itemize
\end{itemize}
& LO1, LO2, LO5, LO7 \\
\hline
Design & Designing architecture and a system that is capable of solving the identified problems with recommended techniques.
\begin{itemize}
\item \textbf{RO8:} Design a price prediction system to identify the possible increase/ decrease in value of the \Gls{nft}s.
\item \textbf{RO9:} Design an automated flow to match \Gls{nft}s with global social trends data.
\item \textbf{RO10:} Design a data-preprocessing pipeline to add Smart Contract data related to \Gls{nft}s in the system.
\item \textbf{RO11:} Design a \Gls{dl} or \Gls{ml} Recommendations model that is capable of appropriately utilizing feature-enhanced data to produce recommendations.
\vspace{-7mm}       % remove line spacing after itemize
\end{itemize}
& LO1 \\
\hline
Development & Implementing a system that is capable of addressing the gaps that were aimed to be solved. 
\begin{itemize}
\item \textbf{RO12:} Develop a Recommendations System that can produce relevant, timely \& trending NFTs (items).
\item \textbf{RO13:} Integrate automation steps in the prototype to enhance features of NFT records and use them to recommend suitable NFTs.
\item \textbf{RO14:} Develop an algorithm that can utilize factors that are considered to affect the desirability of owning an NFT by a person.
\vspace{-7mm}       % remove line spacing after itemize
\end{itemize}
& LO1, LO5, LO6 \\
\hline
Testing and Evaluation & Testing the created system \& Data science models with appropriate data and evaluating them with baseline techniques identified in the literature. 
\begin{itemize}
\item \textbf{RO15:} Create a test plan and perform unit, integration and functional testing.
\item \textbf{RO16:} Evaluate the novel model by bench-marking with  \Gls{p@k} score, compared against baseline models.
\vspace{-7mm}       % remove line spacing after itemize
\end{itemize}
& LO4 \\
\hline
Documenting the progress of the research & Documenting and notifying the continuous progress of the research project and any faced obstacles. 
& LO8, LO6 \\
\hline
Publish Findings & Produce well-structured documentation/ reports/ papers that critically evaluate the research.
\begin{itemize}
\item \textbf{RO17:} Publishing a review paper on related work.
\item \textbf{RO18:} Publishing evaluation \& testing results identified from the research.
\item \textbf{RO19:} Making the code or models created in the research process available for future advancements in research.
\item \textbf{RO20:} Making any modified data-sets or re-creation strategies available to the public, to train \& test models related to similar use cases of utilized data.
\vspace{-7mm}       % remove line spacing after itemize
\end{itemize}
& LO4, LO8 \\
\hline
\end{longtable}
% \end{table}

\section{Novelty of the Research}

The author's research contribution that highlights the novelty of the research can be identified as follows:

\subsection{Technological Novelty}

A Hybrid Recommendations technique that attempts to use public trends in a way that hasn't been attempted in previous research will be explored in order to facilitate the recommendation of relevant, trending and timely items. Automation of several decision-making steps that a user would otherwise need to go through to find the best possible trade will be integrated into the Recommendations Architecture. It is hypothesized that this novel recommendations architecture will be able to be applied to other items as well to give enhanced recommendations based on trends.

\subsection{Application Novelty in Domain}

Currently there is no research work done regarding the recommendation of \gls{nft} assets.
The information in an NFT that has an effect on a user's desire to be owned will be identified, when attempting to provide suitable recommendations.
Looking at the success of Recommendation Systems across multiple systems for over a decade, it is understood that a Recommendation System would help users identify NFTs that they would be interested in trading. This will in return help in increasing sales on NFT Marketplaces and wider adoption of the technology.


\section{Research Challenge}
NFTs is a new domain, which has very less research done related to preferences and factors considered when purchasing NFTs.
Therefore, it is first important to identify the data points (features) \& external factors that affect the value/ desirability of owning NFTs to suggest trading recommendations of NFTs to a user.

\bigbreak
% add this to challenge?
\textit{"Crypto has a founding tradition of emphasizing freedom and privacy. Maybe because of this prevailing cultural trend, the NFT space does not have many recommender systems."} \autocite{noauthor_what_2020}

NFTs are identified to be more challenging to be recommended to users using traditional recommendation methods due to the uniqueness of each item together with the traditions brought forward with the crypto community.
Similar to cryptocurrencies, it has been identified that NFTs too have an impact on the general public opinion \& trends \autocite{dowling_fertile_2021}.

Currently, available Recommendation Systems haven't had the necessity to consider trends as much as with related to the desirability of owning NFTs. Furthermore, scarcity of items opens another challenge of the inability to keep recommending items that are not available for sale or have already been purchased by an interested buyer. But that alone can't be considered due to the time-tested \& proven baseline recommendation techniques being highly effective in multiple domains. Using the identified factors to be considered, a suitable recommendations architecture needs to be implemented.


\section{Chapter Summary}
This chapter presented the problem with necessary proofs and domain description, the research gap, the research challenge, and the research strategy that is expected to be addressed by the author in the research project presented by this document. The research objectives were mapped to the learning outcomes of the project module in the BSc(Hons) Computer Science undergraduate program of the University of Westminster.