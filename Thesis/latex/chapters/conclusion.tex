\section{Chapter Overview}
This chapter brings the thesis of the covered research to a conclusion by marking the concluding statements of the project. The project's unique contribution to the research community is discussed inline with the project aims \& objectives. The challenges that were encountered, how the author's prior knowledge and the modules of the degree program were used, and the new knowledge and skills developed are documented.


\section{Achievement of Research Aim \& Objectives}

\subsection{Achievement of Aims}
\textit{The aim of this research is to design, develop \& evaluate a novel Recommendation Architecture that will provide relevant, trending, timely, and worthy \gls{nft}s for trading purposes by automating some of the decision making steps that the user would otherwise have to do manually.}

The aim of the research was successfully achieved by desigining, developing \& evaluating a novel Recommendations Architecture to produce relevant, trending, timely. Multiple steps of the process were automated to meet with the requirement. Furthermore, research was conducted to identify what models could be used to recommend worthy \gls{nft}s.

\subsection{Achievement of Objectives}
The achievement of the objectives of the research that were mentioned in Chapter 1 have been marked with each of their completion statuses in the \nameref{tab:research-objectives-status-table} table of \textit{\nameref{appendix:conclusion}}.


\section{Utilization of Knowledge from the Course}

% PP1, PP2, OOP, 
% Algorithms, Applied AI
% SDGP
% Web Design and Development

\vspace{-4mm}
\begin{longtable}{|p{0.25\linewidth}|p{0.68\linewidth}|}
\caption{Utilization of knowledge gained from the course}\\ 
\hline
\textbf{Module(s)} & \textbf{Utilized Knowledge}\endfirsthead 
\hline
Software Development Group Project & From recognizing an issue to designing, developing, and testing a prototype, this module gave the initial spark to work on research \& publish research. \\
\hline
Algorithms: Theory Design and Implementation, Applied AI & The knowledge gained from these modules were extremely important when designing new performant algorithms in this research. They also gave quite a lot of knowledge that was used for data science model development. \\
\hline
Programming Principles 1, 2 \& Object Oriented Programming & These modules laid the foundation to programming design used for documentation \& coding concepts used in development of the project. \\
\hline
Enterprise Application Development & Gave a thorough understanding of design documentation \& standard that need to be maintained in Enterprise applications. \\
\hline
Web Design and Development & The prototype was made with UI/ UX guidlines taught in this module. The foundation laid by HTML, CSS \& JavaScript was also helpful as a foundation to go beyond and build perfomant \& advanced UIs. \\
\hline
\end{longtable}


\section{Use of Existing Skills}
% (What is learned from the course? form or outside? applied to the project)
\begin{itemize}
\item \textbf{Fullstack R\&D Development} - The author completed his internship at Zone24x7 where he got to work on fullstack development of R\&D Big Data projects while getting exposed to cutting-edge technologies.
\item \textbf{Blockchain} - The research required quite a lot of understanding in the domain of Blockchain, since \gls{nft}s are one such application of Blockchain technology. It was important to understand people's thinking patterns \& decision making steps to come up with automation tactics required to produce recommendations. The knowledge for this was gained by the author's involvement in Blockchain \& decentralized systems projects at Niftron, the Blockchain Research Group of IIT which was done in collaboration with 99x technology on Blockchain projects\& personal reading.
\item \textbf{\gls{ml}/ \gls{dl}} - The author self-learnt basic \gls{ml} \& \gls{dl} prior to the start of the final year by watching tutorial videos on Youtube \& Coursera.
\end{itemize}


\section{Use of New Skills}
% What you have learned through the project – Not part of the curriculum) – Technical skills should be given preference
\begin{itemize}
\item \textbf{Recommendation Systems} - The author had no prior experience of working with Recommendation Systems. Multiple methods of building Recommendation Systems were learnt using online freely available  material such as Google \gls{ml} courses, Coursera, Medium, Kaggle \& Youtube.
\item \textbf{Data Engineering/ Data Mining \& Information Retrieval} - The author had to learn rigorous data mining \& information retrieval techniques that were required to find the most important information that could be used for the solution since the quality of pre-processing would directly affect the outputs produced.
% \item \textbf{\gls{nft}s} - The author had to self-explore many possibilities of how \gls{nft}s could be recommended since there was no past research related to such applications of \gls{nft}s.
\item \textbf{\gls{nlp}} - Natural Language Processing was used heavily for data extraction \& matching, especially in the pre-processing steps. Additional tutorials \& blogs were followed to understand these concepts.
\item \textbf{\LaTeX }- Used for clear documentation of all research documents for professional typesetting of lasting value. This includes the project proposal, PSPD, Thesis, 2 research papers \& 1 review paper.
\end{itemize}

\section{Achievement of Learning Outcomes (LOs)}
The achievement of the objectives of the research that were mentioned in Chapter 1 have been marked with each of their completion statuses in the \nameref{tab:achievement-learning-outcomes-table} table of \textit{\nameref{appendix:conclusion}}.


\section{Problems and Challenges Faced}
% Need to mention how did you overcome the problems and challenges

\vspace{-4mm}
\begin{longtable}{|p{0.5\linewidth}|p{0.425\linewidth}|}
\caption{Mitigations to Problems and Challenges Faced}\\ 
\hline
\textbf{Problem/ Challenge} & \textbf{Mitigation}\endfirsthead 
\hline
Low battery \& no internet connectivity caused due to long hours of powercuts for more than one month towards the end of the project. & Worked overnight, at co-working spaces and bought a UPS to power the Wifi router and  required peripherals. \\
\hline
Since the \gls{nft} domain is very new \& there wasn't any research done on \gls{nft}s prior to starting the research, it was extremely difficult to find evaluators for the project. Those who were able to evaluate the project without much difficulty had comparatively low amount of paper-qualifications/ prior research experience. & The author had to explore the domain quite a lot. Evaluations were taken from multiple perspectives. Domain \& project evaluations from domain experts \& enthusiasts who were quite young \& senior researchers who could evaluate the project from the research process \& \gls{ml} perspectives. \\
\hline
Lack of \gls{nft} \& trends data \& rate-limited \gls{api}s with \gls{api}-keys. & 
Scripts were written with time-outs to fetch \& preprocess data. \gls{api} keys were requested and received after multiple requests from OpenSea \& Twitter.\\
\hline
Testing \& evaluating the suggested models was challenging since there was no ground-truth to evaluate the models against. Especially, the trends based model. Testing \& evaluating \gls{recsys} models have known to be unclear/ challenging even based on Literature. & Modified testing \& evaluation were conducted for possible models, while algorithmic testing \& evaluation was conducted for the Trends based model. \\
\hline
The novelty of the research \& specificities of the domain made it pretty much impossible to benchmark the system against existing solution. & 2 of the newly created \gls{ml} models were benchmarked against each other. \\
\hline
\end{longtable}

\section{Deviations}
% Any deviations from the original plan should be mentioned and justified
The initial goals of the author was to integrate price-prediction into the system at least in an insignificant manner, but the available data was not sufficient for this purpose and the amount of time \& effort required to fetch \& pre-process data made it clear that it would be required to be done as a separate project.

After considering possible \gls{dl} methods to build a Recommendations Model, it was understood that the amount of data available was not sufficient to attempt it. Algorithmic \gls{ml} models were created instead.

\section{Limitations of the Research}
% Should be linked with the test output
\begin{itemize}
\item Since this project was one of the very first of its kind, there were very limited number of scientific documents, data \& a clear direction to follow. 
\item  The limited time allocated for the research and the need to spend a lot of time mining \& pre-processing data, constrained the author from working on price-prediction and considering the trading \& value aspect of \gls{nft}s for recommendations.
\item The trends based model is only effective if accurate, diverse descriptions are provided for \gls{nft}s.
\item Personalized recommendations cannot be made with the current trends-based model.
\item The words in hashtags aren't split although received as trends. This decreases the possibility of matching with keywords of items.
\item The system works in a centralized manner.
\item The current system may struggle with large-scale system \& data since pure Python \& Pandas are used for the data science component.
\item The current prototype support only Twitter Trends, but can be expanded to use trends from other social platforms.
\end{itemize}


\section{Future Enhancements}
\begin{itemize}
\item Identify possibilities of sourcing trends from more platforms such as Reddit, Discord, Google \& private forums.
\item Attempt to create a decentralized Recommendations Eco-system using the Trends-based \gls{recsys} model, since the trends and items can come from two different sources.
\item The current solution does a string match with keywords of each item. This may cause some matches to be skipped due to appearing in different forms. The \gls{nlp} technique, lemmatization could be a possible solution for this. Name Entity Recognition is another \gls{nlp} technique that could enhance the quality of trends data used. The significance of introducing such techniques will have to be tested since they may not have a significant impact on the output as most trends appear to be nouns.
\item One of the short-comings to help match trends for this purpose is that Twitter trends contain hashtags as trends names at times. Either the developers from the end of Twitter could give a possible solution to it or hashtags may have to be pre-processed and separated.
\item Work on a price-prediction model for \gls{nft}s. This may be extremely difficult due to the uniqueness of \gls{nft}s and due to the low amount of available data. A suggestion that I received for this was to attempt using a dataset of rare-physical artwork since they tend to resemble the nature of \gls{nft} pricing.
\textit{Closer to the completion of this research, the author came across a \gls{nft} dataset \autocite{zomglings_ethereum_2021} that may be usable for price/ bid prediction training purposes.}
\item As a substitute or addition to recently released movies, Amazon's \gls{dl} Neural Network Model could make use of trends, maybe to bolster recommendations for movies as well as e-commerce items. Due to the lack of \gls{nft} data, this \gls{dl} based approach could not be attempted.
\item The trends could be categorized to identify similar trends that users seem to show interest in. It would be almost impossible to attempt this level of personalization without collecting user data. Therefore, the value of such an attempt may have to be justified.
\end{itemize}


\section{Achievement of the Contribution to Body of Knowledge}
%  explain in para - maybe highlight points?

By concluding the research project, the author has managed to make contributions in the domain of \gls{nft}s, technology of Recommendation Systems and towards the research process.

\subsection{Technical Contribution (Recommendations Systems)}

\begin{enumerate}
\item Social Trends influenced Recommendation Model - a novel \& innovative concept \& approach taken.
\item Trends-score calculation equation \& algorithm
\end{enumerate}

\noindent Currently, no Sri Lankan based e-commerce site currently provides recommendations. The introduced Trends-based \gls{recsys} will help generate timely \& trending recommendations without having to collect \& store large amounts of user-data.

\subsection{Domain Contribution (NFTs)}
Identification \& analysis of factors that can be used to produce \gls{nft} item recommendations.

\subsection{Additional Contributions}
\begin{enumerate}
\item Data Preprocessing scripts
    \begin{enumerate}
        \item Social media trends extraction
        \item NFT Data extraction
        \begin{itemize}
            \item Trait rarity calculation
        \end{itemize}
     \end{enumerate}
     
\item NFT Asset datasets
\item Created a Latex template for the expected thesis structure that can be used by IIT students in the future (\textit{Contributions made by Visal Rajapakse, Isala Piyarisi \& Akassharjun Shanmugarajah}).
\end{enumerate}


\section{Concluding Remarks}
This concludes a research which lays the ground work for future \gls{nft} research projects \& novel ways of generating recommendations using sparse data. The introduced algorithm has the potential to evolve into a much complex \& utilitarian Recommendation System which may be embraced by digital systems \& internet of the next decade.
The research process was conducted at the highest possible standards signifying the contributions made to the body of knowledge.